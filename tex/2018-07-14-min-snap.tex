\documentclass{article}
\usepackage{hyperref}
\hypersetup{
    colorlinks=true,
    linkcolor=blue,
    filecolor=magenta,      
    urlcolor=cyan,
}
\usepackage{amsfonts}
\usepackage{mathtools}
\let\vec\mathbf
\newcommand{\norm}[1]{\left\lVert#1\right\rVert}

% Probability commands: Expectation, Variance, Covariance, Bias
\newcommand{\E}{\mathrm{E}}
\newcommand{\Var}{\mathrm{Var}}
\newcommand{\Cov}{\mathrm{Cov}}
\newcommand{\Bias}{\mathrm{Bias}}


\begin{document}
\section{Introduction}
Back when I was doing my masters degree, my research advisor suggested I take a numerical optimization class. I was a bit reluctant to do so because of my very non-mathematical background (Computer Engineering) but I knew it was important if I wanted to get into the field of robotics. Since the class required me to implement a project, I decided to work on one of the most important papers in the world of multirotors: \href{https://ieeexplore.ieee.org/document/5980409/}{Minimum Snap Trajectory Generation and Control for Quadrotors} by Daniel Mellinger and Vijay Kumar. In this article, I will attempt to explain it for people who whish to understand and reimplement the method in a way that can be understood by readers without a strong control theory background (like me). If you have a background in systems and controls, mechanical engineering or the like, then this post is probably too verbose for you.

The importance of the contribution of Mellinger and Kumar is two fold, first they prove the differential flatness of a quadrotor system and second they show a method for trajectory generation for quadrotors. A system is differentially flat if you can express its states and inputs as a function of its outputs and the derivatives of these. Mellinger chooses the states of the quadrotor to be

\[
	\mathbf{x} = [ x, y, z, \phi, \theta, \psi, \dot{x}, \dot{y}, \dot{z}, p, q, r]^\top
\]

In other words, the states are the position $[x, y, z]$, the orientation $[\phi, \theta, \psi]$ (roll, pitch, yaw), the velocity $[\dot{x}, \dot{y}, \dot{z}]$ and the angular velocity $[p, q, r]$ and the flat outputs is chosen to be

\[
	\sigma = [x, y, z, \psi]^\top
\]

and our inputs are the body moments $\mathbf{•}{u}$ which can be directly mapped to the rotor speeds of your quadrotors. Finally, we define a trajectory as

\[
	\sigma (t) : [t_0, t_m] \rightarrow  \mathbb{R}^3 \times SO(2)
\]

i.e. a trajectory in 3D space and yaw space defined from time $ t_0 $ to $t_m$. Practically speaking, it is useful to define a trajectory through a series of waypoints (Mellinger uses the word "keyframe") by a series of piecewise defined polynomials of order $n$ over $m$ intervals.

\begin{align}\label{eq:polynomial}
\sigma_T(t) =
\left\{
	\begin{array}{ll}
		\sum_{i=0}^n \sigma_{Ti1} t^i  & t_0 \leq t < t_1 \\
		\sum_{i=0}^n \sigma_{Ti2} t^i  & t_1 \leq t < t_2 \\
		... \\
		\sum_{i=0}^n \sigma_{Tim} t^i  & t_{m-1} \leq t < t_m \\
	\end{array}
\right.
\end{align}

\section{Trajectory Generation}

Let's now examine how we can formulate a trajectory generation problem knowing the previous information. Mellinger explains that we want to minimize ``the integral of the squared sum of the $k_r$th and the $k_\psi$th derivative of yaw angle squared''.
\[
\text{min} \int_{t_0}^{t_m} \mu_r \norm{\frac{d^{k_r} r_T}{dt^{k_r}}}^2 + \mu_\psi \frac{d^{k_\psi} \psi_T}{dt^{k_\psi}}^2
\]
\[
	\begin{array}{lll}
		\text{subject to} & \sigma_T(t_i) = \sigma_i & i = 0, \ldots, m\\
		& \frac{d^p x_T}{dt^p}|_{t=t_j} = 0\ \text{or free,} & j = 0, \ldots, m; p = 1, \ldots, k_r\\
		& \frac{d^p y_T}{dt^p}|_{t=t_j} = 0\ \text{or free,} & j = 0, \ldots, m; p = 1, \ldots, k_r\\
		& \frac{d^p z_T}{dt^p}|_{t=t_j} = 0\ \text{or free,} & j = 0, \ldots, m; p = 1, \ldots, k_r\\
		& \frac{d^p \psi_T}{dt^p}|_{t=t_j} = 0\ \text{or free,} & j = 0, \ldots, m; p = 1, \ldots, k_\psi\\
	\end{array}
\]
where $\mu_r$ and $\mu_\psi$ are constants that make the integrand nondimensional. Before we move on, Melling explains a most interesting property of our polynomial trajectorties called non-dimensionalization.

\subsection{Why Minimize Snap?}
You might be wondering, why are we minimizing snap specifically rather than any other of the derivatives? 

\begin{quotation}
In our system, since the inputs $u_2$ and $u_3$ appear as functions of the fourth derivatives of the positions, we generate trajectories that minimize the integral of the square of the norm of the snap (the second derivative of acceleration, $k_r = 4$).
\end{quotation}

In other words we minimize the snap because the quadrotor's roll and pitch are directly related to the trajectory's snap. This also means that we will end up with a trajectory where the roll and pitch will be continuous, which would be nice for our vehicle to be able to fly it :).

\subsection{Formulating a QP Problem}
If you aren't familiar with optimization theory, quadratic programming is the process of solving an optimization problem of the form 

\[
\text{min}\ \ \ c^THc+f^Tc
\]\[
	\begin{array}{ll}
	\text{subject to} & Ac\leq b
	\end{array}
\]

where your vector $c$ is filled with your decision variables and you use matrix $A$ and vector $b$ to encode your equality (or inequality) constraints. In our case $c$ will be the constant coefficients of our polynomials $\sigma_{T_{ij}} = [x_{T_{ij}}, y_{T_{ij}}, z_{T_{ij}}, \psi_{T_{ij}}]$ and will be of size $4nm \times 1$.

\end{document} 